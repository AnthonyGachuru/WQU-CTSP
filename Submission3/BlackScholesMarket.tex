\documentclass{article}
\usepackage{graphicx}
% stuff from the percent sign to end of line is a comment, ignored by LaTeX

\usepackage{amsmath,amssymb,graphicx} %load extra symbols and environments
\usepackage[margin=1in]{geometry} %set margins
\usepackage{enumerate}
\begin{document}

\nocite{*} % this command forces all references in template.bib to be printed in the bibliography

\title{Short Rate Models}

\author{
  Ansari, Zain Us Sami Ahmed\\
  \texttt{zainussami@gmail.com}
  \and
  Nguyen, Dang Duy Nghia \\
  \texttt{nghia002@e.ntu.edu.sg}  
  }

\date{March. 5, 2020} % if this is omitted, the current date is used for the title page

\maketitle

\noindent
\textbf{Keywords:} Short rate models, Interest Rates Derivatives,  Ho-Lee Model, Vasicek Model, Cox-Ingersoll-Ross (CIR) model, Single factor and two factor Hull-White model.



% the following creates an abstract -- it can be omitted
% an example of an environment: these have the form \begin{name} ... \end{name}
\begin{abstract}
In this paper we discuss the interest rate models with a focus on two factor Hull-White model.  In the first section , in the second section we describe the two factor Hull-White model in mathematical terms, in the third section we compare this model to Ho-Lee Model, Vasicek Model, Cox-Ingersoll-Ross (CIR) model, Single factor Hull-White model,  in the fourth section we present a derivative product payoff using the two factor Hull-White model. and we conclude by discussing the purpose of this paper.  
\end{abstract}

\section{Introduction
}




\section{Two-Factor Hull-White Model 
}


\section{Comparison with Single Factor Short Rate Models
}

\section{Derivative product Payoff}


\section{Conclusion}

In this paper we went through ....................  The purpose of this paper ..............................

\newpage
\section*{} \label{bibsection}


% the second parameter MMMMM should be as long as the longest label you use, in my case Smoller -- if you use % numbers only, use 99
% use \cite{refname} to refer to bibliography item \bibitem{refname} 
% LaTeX assigns a number, unless you use \bibitem[Name]{refname} -- in this case
% LaTeX prints Name when you use \cite{refname}
\begin{thebibliography}{MMMMM} 
\bibitem{BA1} Blanchard, Arnaud. "The two-factor Hull-White model: pricing and calibration of interest rates derivatives." KTH Royal Insitute of Technology (2014).


\end{thebibliography}
\bibliographystyle{plain}
\bibliography{template}

\end{document}
