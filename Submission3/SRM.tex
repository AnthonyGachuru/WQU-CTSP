\documentclass[11pt]{article}

    \usepackage[breakable]{tcolorbox}
    \usepackage{parskip} % Stop auto-indenting (to mimic markdown behaviour)
    
    \usepackage{iftex}
    \ifPDFTeX
    	\usepackage[T1]{fontenc}
    	\usepackage{mathpazo}
    \else
    	\usepackage{fontspec}
    \fi

    % Basic figure setup, for now with no caption control since it's done
    % automatically by Pandoc (which extracts ![](path) syntax from Markdown).
    \usepackage{graphicx}
    % Maintain compatibility with old templates. Remove in nbconvert 6.0
    \let\Oldincludegraphics\includegraphics
    % Ensure that by default, figures have no caption (until we provide a
    % proper Figure object with a Caption API and a way to capture that
    % in the conversion process - todo).
    \usepackage{caption}
    \DeclareCaptionFormat{nocaption}{}
    \captionsetup{format=nocaption,aboveskip=0pt,belowskip=0pt}

    \usepackage[Export]{adjustbox} % Used to constrain images to a maximum size
    \adjustboxset{max size={0.9\linewidth}{0.9\paperheight}}
    \usepackage{float}
    \floatplacement{figure}{H} % forces figures to be placed at the correct location
    \usepackage{xcolor} % Allow colors to be defined
    \usepackage{enumerate} % Needed for markdown enumerations to work
    \usepackage{geometry} % Used to adjust the document margins
    \usepackage{amsmath} % Equations
    \usepackage{amssymb} % Equations
    \usepackage{textcomp} % defines textquotesingle
    % Hack from http://tex.stackexchange.com/a/47451/13684:
    \AtBeginDocument{%
        \def\PYZsq{\textquotesingle}% Upright quotes in Pygmentized code
    }
    \usepackage{upquote} % Upright quotes for verbatim code
    \usepackage{eurosym} % defines \euro
    \usepackage[mathletters]{ucs} % Extended unicode (utf-8) support
    \usepackage{fancyvrb} % verbatim replacement that allows latex
    \usepackage{grffile} % extends the file name processing of package graphics 
                         % to support a larger range
    \makeatletter % fix for grffile with XeLaTeX
    \def\Gread@@xetex#1{%
      \IfFileExists{"\Gin@base".bb}%
      {\Gread@eps{\Gin@base.bb}}%
      {\Gread@@xetex@aux#1}%
    }
    \makeatother

    % The hyperref package gives us a pdf with properly built
    % internal navigation ('pdf bookmarks' for the table of contents,
    % internal cross-reference links, web links for URLs, etc.)
    \usepackage{hyperref}
    % The default LaTeX title has an obnoxious amount of whitespace. By default,
    % titling removes some of it. It also provides customization options.
    \usepackage{titling}
    \usepackage{longtable} % longtable support required by pandoc >1.10
    \usepackage{booktabs}  % table support for pandoc > 1.12.2
    \usepackage[inline]{enumitem} % IRkernel/repr support (it uses the enumerate* environment)
    \usepackage[normalem]{ulem} % ulem is needed to support strikethroughs (\sout)
                                % normalem makes italics be italics, not underlines
    \usepackage{mathrsfs}
    

    
    % Colors for the hyperref package
    \definecolor{urlcolor}{rgb}{0,.145,.698}
    \definecolor{linkcolor}{rgb}{.71,0.21,0.01}
    \definecolor{citecolor}{rgb}{.12,.54,.11}

    % ANSI colors
    \definecolor{ansi-black}{HTML}{3E424D}
    \definecolor{ansi-black-intense}{HTML}{282C36}
    \definecolor{ansi-red}{HTML}{E75C58}
    \definecolor{ansi-red-intense}{HTML}{B22B31}
    \definecolor{ansi-green}{HTML}{00A250}
    \definecolor{ansi-green-intense}{HTML}{007427}
    \definecolor{ansi-yellow}{HTML}{DDB62B}
    \definecolor{ansi-yellow-intense}{HTML}{B27D12}
    \definecolor{ansi-blue}{HTML}{208FFB}
    \definecolor{ansi-blue-intense}{HTML}{0065CA}
    \definecolor{ansi-magenta}{HTML}{D160C4}
    \definecolor{ansi-magenta-intense}{HTML}{A03196}
    \definecolor{ansi-cyan}{HTML}{60C6C8}
    \definecolor{ansi-cyan-intense}{HTML}{258F8F}
    \definecolor{ansi-white}{HTML}{C5C1B4}
    \definecolor{ansi-white-intense}{HTML}{A1A6B2}
    \definecolor{ansi-default-inverse-fg}{HTML}{FFFFFF}
    \definecolor{ansi-default-inverse-bg}{HTML}{000000}

    % commands and environments needed by pandoc snippets
    % extracted from the output of `pandoc -s`
    \providecommand{\tightlist}{%
      \setlength{\itemsep}{0pt}\setlength{\parskip}{0pt}}
    \DefineVerbatimEnvironment{Highlighting}{Verbatim}{commandchars=\\\{\}}
    % Add ',fontsize=\small' for more characters per line
    \newenvironment{Shaded}{}{}
    \newcommand{\KeywordTok}[1]{\textcolor[rgb]{0.00,0.44,0.13}{\textbf{{#1}}}}
    \newcommand{\DataTypeTok}[1]{\textcolor[rgb]{0.56,0.13,0.00}{{#1}}}
    \newcommand{\DecValTok}[1]{\textcolor[rgb]{0.25,0.63,0.44}{{#1}}}
    \newcommand{\BaseNTok}[1]{\textcolor[rgb]{0.25,0.63,0.44}{{#1}}}
    \newcommand{\FloatTok}[1]{\textcolor[rgb]{0.25,0.63,0.44}{{#1}}}
    \newcommand{\CharTok}[1]{\textcolor[rgb]{0.25,0.44,0.63}{{#1}}}
    \newcommand{\StringTok}[1]{\textcolor[rgb]{0.25,0.44,0.63}{{#1}}}
    \newcommand{\CommentTok}[1]{\textcolor[rgb]{0.38,0.63,0.69}{\textit{{#1}}}}
    \newcommand{\OtherTok}[1]{\textcolor[rgb]{0.00,0.44,0.13}{{#1}}}
    \newcommand{\AlertTok}[1]{\textcolor[rgb]{1.00,0.00,0.00}{\textbf{{#1}}}}
    \newcommand{\FunctionTok}[1]{\textcolor[rgb]{0.02,0.16,0.49}{{#1}}}
    \newcommand{\RegionMarkerTok}[1]{{#1}}
    \newcommand{\ErrorTok}[1]{\textcolor[rgb]{1.00,0.00,0.00}{\textbf{{#1}}}}
    \newcommand{\NormalTok}[1]{{#1}}
    
    % Additional commands for more recent versions of Pandoc
    \newcommand{\ConstantTok}[1]{\textcolor[rgb]{0.53,0.00,0.00}{{#1}}}
    \newcommand{\SpecialCharTok}[1]{\textcolor[rgb]{0.25,0.44,0.63}{{#1}}}
    \newcommand{\VerbatimStringTok}[1]{\textcolor[rgb]{0.25,0.44,0.63}{{#1}}}
    \newcommand{\SpecialStringTok}[1]{\textcolor[rgb]{0.73,0.40,0.53}{{#1}}}
    \newcommand{\ImportTok}[1]{{#1}}
    \newcommand{\DocumentationTok}[1]{\textcolor[rgb]{0.73,0.13,0.13}{\textit{{#1}}}}
    \newcommand{\AnnotationTok}[1]{\textcolor[rgb]{0.38,0.63,0.69}{\textbf{\textit{{#1}}}}}
    \newcommand{\CommentVarTok}[1]{\textcolor[rgb]{0.38,0.63,0.69}{\textbf{\textit{{#1}}}}}
    \newcommand{\VariableTok}[1]{\textcolor[rgb]{0.10,0.09,0.49}{{#1}}}
    \newcommand{\ControlFlowTok}[1]{\textcolor[rgb]{0.00,0.44,0.13}{\textbf{{#1}}}}
    \newcommand{\OperatorTok}[1]{\textcolor[rgb]{0.40,0.40,0.40}{{#1}}}
    \newcommand{\BuiltInTok}[1]{{#1}}
    \newcommand{\ExtensionTok}[1]{{#1}}
    \newcommand{\PreprocessorTok}[1]{\textcolor[rgb]{0.74,0.48,0.00}{{#1}}}
    \newcommand{\AttributeTok}[1]{\textcolor[rgb]{0.49,0.56,0.16}{{#1}}}
    \newcommand{\InformationTok}[1]{\textcolor[rgb]{0.38,0.63,0.69}{\textbf{\textit{{#1}}}}}
    \newcommand{\WarningTok}[1]{\textcolor[rgb]{0.38,0.63,0.69}{\textbf{\textit{{#1}}}}}
    
    
    % Define a nice break command that doesn't care if a line doesn't already
    % exist.
    \def\br{\hspace*{\fill} \\* }
    % Math Jax compatibility definitions
    \def\gt{>}
    \def\lt{<}
    \let\Oldtex\TeX
    \let\Oldlatex\LaTeX
    \renewcommand{\TeX}{\textrm{\Oldtex}}
    \renewcommand{\LaTeX}{\textrm{\Oldlatex}}
    % Document parameters
    % Document title
    \title{2-factor-Hull-White\_model\_calibration}
    
    
    
    
    
% Pygments definitions
\makeatletter
\def\PY@reset{\let\PY@it=\relax \let\PY@bf=\relax%
    \let\PY@ul=\relax \let\PY@tc=\relax%
    \let\PY@bc=\relax \let\PY@ff=\relax}
\def\PY@tok#1{\csname PY@tok@#1\endcsname}
\def\PY@toks#1+{\ifx\relax#1\empty\else%
    \PY@tok{#1}\expandafter\PY@toks\fi}
\def\PY@do#1{\PY@bc{\PY@tc{\PY@ul{%
    \PY@it{\PY@bf{\PY@ff{#1}}}}}}}
\def\PY#1#2{\PY@reset\PY@toks#1+\relax+\PY@do{#2}}

\expandafter\def\csname PY@tok@w\endcsname{\def\PY@tc##1{\textcolor[rgb]{0.73,0.73,0.73}{##1}}}
\expandafter\def\csname PY@tok@c\endcsname{\let\PY@it=\textit\def\PY@tc##1{\textcolor[rgb]{0.25,0.50,0.50}{##1}}}
\expandafter\def\csname PY@tok@cp\endcsname{\def\PY@tc##1{\textcolor[rgb]{0.74,0.48,0.00}{##1}}}
\expandafter\def\csname PY@tok@k\endcsname{\let\PY@bf=\textbf\def\PY@tc##1{\textcolor[rgb]{0.00,0.50,0.00}{##1}}}
\expandafter\def\csname PY@tok@kp\endcsname{\def\PY@tc##1{\textcolor[rgb]{0.00,0.50,0.00}{##1}}}
\expandafter\def\csname PY@tok@kt\endcsname{\def\PY@tc##1{\textcolor[rgb]{0.69,0.00,0.25}{##1}}}
\expandafter\def\csname PY@tok@o\endcsname{\def\PY@tc##1{\textcolor[rgb]{0.40,0.40,0.40}{##1}}}
\expandafter\def\csname PY@tok@ow\endcsname{\let\PY@bf=\textbf\def\PY@tc##1{\textcolor[rgb]{0.67,0.13,1.00}{##1}}}
\expandafter\def\csname PY@tok@nb\endcsname{\def\PY@tc##1{\textcolor[rgb]{0.00,0.50,0.00}{##1}}}
\expandafter\def\csname PY@tok@nf\endcsname{\def\PY@tc##1{\textcolor[rgb]{0.00,0.00,1.00}{##1}}}
\expandafter\def\csname PY@tok@nc\endcsname{\let\PY@bf=\textbf\def\PY@tc##1{\textcolor[rgb]{0.00,0.00,1.00}{##1}}}
\expandafter\def\csname PY@tok@nn\endcsname{\let\PY@bf=\textbf\def\PY@tc##1{\textcolor[rgb]{0.00,0.00,1.00}{##1}}}
\expandafter\def\csname PY@tok@ne\endcsname{\let\PY@bf=\textbf\def\PY@tc##1{\textcolor[rgb]{0.82,0.25,0.23}{##1}}}
\expandafter\def\csname PY@tok@nv\endcsname{\def\PY@tc##1{\textcolor[rgb]{0.10,0.09,0.49}{##1}}}
\expandafter\def\csname PY@tok@no\endcsname{\def\PY@tc##1{\textcolor[rgb]{0.53,0.00,0.00}{##1}}}
\expandafter\def\csname PY@tok@nl\endcsname{\def\PY@tc##1{\textcolor[rgb]{0.63,0.63,0.00}{##1}}}
\expandafter\def\csname PY@tok@ni\endcsname{\let\PY@bf=\textbf\def\PY@tc##1{\textcolor[rgb]{0.60,0.60,0.60}{##1}}}
\expandafter\def\csname PY@tok@na\endcsname{\def\PY@tc##1{\textcolor[rgb]{0.49,0.56,0.16}{##1}}}
\expandafter\def\csname PY@tok@nt\endcsname{\let\PY@bf=\textbf\def\PY@tc##1{\textcolor[rgb]{0.00,0.50,0.00}{##1}}}
\expandafter\def\csname PY@tok@nd\endcsname{\def\PY@tc##1{\textcolor[rgb]{0.67,0.13,1.00}{##1}}}
\expandafter\def\csname PY@tok@s\endcsname{\def\PY@tc##1{\textcolor[rgb]{0.73,0.13,0.13}{##1}}}
\expandafter\def\csname PY@tok@sd\endcsname{\let\PY@it=\textit\def\PY@tc##1{\textcolor[rgb]{0.73,0.13,0.13}{##1}}}
\expandafter\def\csname PY@tok@si\endcsname{\let\PY@bf=\textbf\def\PY@tc##1{\textcolor[rgb]{0.73,0.40,0.53}{##1}}}
\expandafter\def\csname PY@tok@se\endcsname{\let\PY@bf=\textbf\def\PY@tc##1{\textcolor[rgb]{0.73,0.40,0.13}{##1}}}
\expandafter\def\csname PY@tok@sr\endcsname{\def\PY@tc##1{\textcolor[rgb]{0.73,0.40,0.53}{##1}}}
\expandafter\def\csname PY@tok@ss\endcsname{\def\PY@tc##1{\textcolor[rgb]{0.10,0.09,0.49}{##1}}}
\expandafter\def\csname PY@tok@sx\endcsname{\def\PY@tc##1{\textcolor[rgb]{0.00,0.50,0.00}{##1}}}
\expandafter\def\csname PY@tok@m\endcsname{\def\PY@tc##1{\textcolor[rgb]{0.40,0.40,0.40}{##1}}}
\expandafter\def\csname PY@tok@gh\endcsname{\let\PY@bf=\textbf\def\PY@tc##1{\textcolor[rgb]{0.00,0.00,0.50}{##1}}}
\expandafter\def\csname PY@tok@gu\endcsname{\let\PY@bf=\textbf\def\PY@tc##1{\textcolor[rgb]{0.50,0.00,0.50}{##1}}}
\expandafter\def\csname PY@tok@gd\endcsname{\def\PY@tc##1{\textcolor[rgb]{0.63,0.00,0.00}{##1}}}
\expandafter\def\csname PY@tok@gi\endcsname{\def\PY@tc##1{\textcolor[rgb]{0.00,0.63,0.00}{##1}}}
\expandafter\def\csname PY@tok@gr\endcsname{\def\PY@tc##1{\textcolor[rgb]{1.00,0.00,0.00}{##1}}}
\expandafter\def\csname PY@tok@ge\endcsname{\let\PY@it=\textit}
\expandafter\def\csname PY@tok@gs\endcsname{\let\PY@bf=\textbf}
\expandafter\def\csname PY@tok@gp\endcsname{\let\PY@bf=\textbf\def\PY@tc##1{\textcolor[rgb]{0.00,0.00,0.50}{##1}}}
\expandafter\def\csname PY@tok@go\endcsname{\def\PY@tc##1{\textcolor[rgb]{0.53,0.53,0.53}{##1}}}
\expandafter\def\csname PY@tok@gt\endcsname{\def\PY@tc##1{\textcolor[rgb]{0.00,0.27,0.87}{##1}}}
\expandafter\def\csname PY@tok@err\endcsname{\def\PY@bc##1{\setlength{\fboxsep}{0pt}\fcolorbox[rgb]{1.00,0.00,0.00}{1,1,1}{\strut ##1}}}
\expandafter\def\csname PY@tok@kc\endcsname{\let\PY@bf=\textbf\def\PY@tc##1{\textcolor[rgb]{0.00,0.50,0.00}{##1}}}
\expandafter\def\csname PY@tok@kd\endcsname{\let\PY@bf=\textbf\def\PY@tc##1{\textcolor[rgb]{0.00,0.50,0.00}{##1}}}
\expandafter\def\csname PY@tok@kn\endcsname{\let\PY@bf=\textbf\def\PY@tc##1{\textcolor[rgb]{0.00,0.50,0.00}{##1}}}
\expandafter\def\csname PY@tok@kr\endcsname{\let\PY@bf=\textbf\def\PY@tc##1{\textcolor[rgb]{0.00,0.50,0.00}{##1}}}
\expandafter\def\csname PY@tok@bp\endcsname{\def\PY@tc##1{\textcolor[rgb]{0.00,0.50,0.00}{##1}}}
\expandafter\def\csname PY@tok@fm\endcsname{\def\PY@tc##1{\textcolor[rgb]{0.00,0.00,1.00}{##1}}}
\expandafter\def\csname PY@tok@vc\endcsname{\def\PY@tc##1{\textcolor[rgb]{0.10,0.09,0.49}{##1}}}
\expandafter\def\csname PY@tok@vg\endcsname{\def\PY@tc##1{\textcolor[rgb]{0.10,0.09,0.49}{##1}}}
\expandafter\def\csname PY@tok@vi\endcsname{\def\PY@tc##1{\textcolor[rgb]{0.10,0.09,0.49}{##1}}}
\expandafter\def\csname PY@tok@vm\endcsname{\def\PY@tc##1{\textcolor[rgb]{0.10,0.09,0.49}{##1}}}
\expandafter\def\csname PY@tok@sa\endcsname{\def\PY@tc##1{\textcolor[rgb]{0.73,0.13,0.13}{##1}}}
\expandafter\def\csname PY@tok@sb\endcsname{\def\PY@tc##1{\textcolor[rgb]{0.73,0.13,0.13}{##1}}}
\expandafter\def\csname PY@tok@sc\endcsname{\def\PY@tc##1{\textcolor[rgb]{0.73,0.13,0.13}{##1}}}
\expandafter\def\csname PY@tok@dl\endcsname{\def\PY@tc##1{\textcolor[rgb]{0.73,0.13,0.13}{##1}}}
\expandafter\def\csname PY@tok@s2\endcsname{\def\PY@tc##1{\textcolor[rgb]{0.73,0.13,0.13}{##1}}}
\expandafter\def\csname PY@tok@sh\endcsname{\def\PY@tc##1{\textcolor[rgb]{0.73,0.13,0.13}{##1}}}
\expandafter\def\csname PY@tok@s1\endcsname{\def\PY@tc##1{\textcolor[rgb]{0.73,0.13,0.13}{##1}}}
\expandafter\def\csname PY@tok@mb\endcsname{\def\PY@tc##1{\textcolor[rgb]{0.40,0.40,0.40}{##1}}}
\expandafter\def\csname PY@tok@mf\endcsname{\def\PY@tc##1{\textcolor[rgb]{0.40,0.40,0.40}{##1}}}
\expandafter\def\csname PY@tok@mh\endcsname{\def\PY@tc##1{\textcolor[rgb]{0.40,0.40,0.40}{##1}}}
\expandafter\def\csname PY@tok@mi\endcsname{\def\PY@tc##1{\textcolor[rgb]{0.40,0.40,0.40}{##1}}}
\expandafter\def\csname PY@tok@il\endcsname{\def\PY@tc##1{\textcolor[rgb]{0.40,0.40,0.40}{##1}}}
\expandafter\def\csname PY@tok@mo\endcsname{\def\PY@tc##1{\textcolor[rgb]{0.40,0.40,0.40}{##1}}}
\expandafter\def\csname PY@tok@ch\endcsname{\let\PY@it=\textit\def\PY@tc##1{\textcolor[rgb]{0.25,0.50,0.50}{##1}}}
\expandafter\def\csname PY@tok@cm\endcsname{\let\PY@it=\textit\def\PY@tc##1{\textcolor[rgb]{0.25,0.50,0.50}{##1}}}
\expandafter\def\csname PY@tok@cpf\endcsname{\let\PY@it=\textit\def\PY@tc##1{\textcolor[rgb]{0.25,0.50,0.50}{##1}}}
\expandafter\def\csname PY@tok@c1\endcsname{\let\PY@it=\textit\def\PY@tc##1{\textcolor[rgb]{0.25,0.50,0.50}{##1}}}
\expandafter\def\csname PY@tok@cs\endcsname{\let\PY@it=\textit\def\PY@tc##1{\textcolor[rgb]{0.25,0.50,0.50}{##1}}}

\def\PYZbs{\char`\\}
\def\PYZus{\char`\_}
\def\PYZob{\char`\{}
\def\PYZcb{\char`\}}
\def\PYZca{\char`\^}
\def\PYZam{\char`\&}
\def\PYZlt{\char`\<}
\def\PYZgt{\char`\>}
\def\PYZsh{\char`\#}
\def\PYZpc{\char`\%}
\def\PYZdl{\char`\$}
\def\PYZhy{\char`\-}
\def\PYZsq{\char`\'}
\def\PYZdq{\char`\"}
\def\PYZti{\char`\~}
% for compatibility with earlier versions
\def\PYZat{@}
\def\PYZlb{[}
\def\PYZrb{]}
\makeatother


    % For linebreaks inside Verbatim environment from package fancyvrb. 
    \makeatletter
        \newbox\Wrappedcontinuationbox 
        \newbox\Wrappedvisiblespacebox 
        \newcommand*\Wrappedvisiblespace {\textcolor{red}{\textvisiblespace}} 
        \newcommand*\Wrappedcontinuationsymbol {\textcolor{red}{\llap{\tiny$\m@th\hookrightarrow$}}} 
        \newcommand*\Wrappedcontinuationindent {3ex } 
        \newcommand*\Wrappedafterbreak {\kern\Wrappedcontinuationindent\copy\Wrappedcontinuationbox} 
        % Take advantage of the already applied Pygments mark-up to insert 
        % potential linebreaks for TeX processing. 
        %        {, <, #, %, $, ' and ": go to next line. 
        %        _, }, ^, &, >, - and ~: stay at end of broken line. 
        % Use of \textquotesingle for straight quote. 
        \newcommand*\Wrappedbreaksatspecials {% 
            \def\PYGZus{\discretionary{\char`\_}{\Wrappedafterbreak}{\char`\_}}% 
            \def\PYGZob{\discretionary{}{\Wrappedafterbreak\char`\{}{\char`\{}}% 
            \def\PYGZcb{\discretionary{\char`\}}{\Wrappedafterbreak}{\char`\}}}% 
            \def\PYGZca{\discretionary{\char`\^}{\Wrappedafterbreak}{\char`\^}}% 
            \def\PYGZam{\discretionary{\char`\&}{\Wrappedafterbreak}{\char`\&}}% 
            \def\PYGZlt{\discretionary{}{\Wrappedafterbreak\char`\<}{\char`\<}}% 
            \def\PYGZgt{\discretionary{\char`\>}{\Wrappedafterbreak}{\char`\>}}% 
            \def\PYGZsh{\discretionary{}{\Wrappedafterbreak\char`\#}{\char`\#}}% 
            \def\PYGZpc{\discretionary{}{\Wrappedafterbreak\char`\%}{\char`\%}}% 
            \def\PYGZdl{\discretionary{}{\Wrappedafterbreak\char`\$}{\char`\$}}% 
            \def\PYGZhy{\discretionary{\char`\-}{\Wrappedafterbreak}{\char`\-}}% 
            \def\PYGZsq{\discretionary{}{\Wrappedafterbreak\textquotesingle}{\textquotesingle}}% 
            \def\PYGZdq{\discretionary{}{\Wrappedafterbreak\char`\"}{\char`\"}}% 
            \def\PYGZti{\discretionary{\char`\~}{\Wrappedafterbreak}{\char`\~}}% 
        } 
        % Some characters . , ; ? ! / are not pygmentized. 
        % This macro makes them "active" and they will insert potential linebreaks 
        \newcommand*\Wrappedbreaksatpunct {% 
            \lccode`\~`\.\lowercase{\def~}{\discretionary{\hbox{\char`\.}}{\Wrappedafterbreak}{\hbox{\char`\.}}}% 
            \lccode`\~`\,\lowercase{\def~}{\discretionary{\hbox{\char`\,}}{\Wrappedafterbreak}{\hbox{\char`\,}}}% 
            \lccode`\~`\;\lowercase{\def~}{\discretionary{\hbox{\char`\;}}{\Wrappedafterbreak}{\hbox{\char`\;}}}% 
            \lccode`\~`\:\lowercase{\def~}{\discretionary{\hbox{\char`\:}}{\Wrappedafterbreak}{\hbox{\char`\:}}}% 
            \lccode`\~`\?\lowercase{\def~}{\discretionary{\hbox{\char`\?}}{\Wrappedafterbreak}{\hbox{\char`\?}}}% 
            \lccode`\~`\!\lowercase{\def~}{\discretionary{\hbox{\char`\!}}{\Wrappedafterbreak}{\hbox{\char`\!}}}% 
            \lccode`\~`\/\lowercase{\def~}{\discretionary{\hbox{\char`\/}}{\Wrappedafterbreak}{\hbox{\char`\/}}}% 
            \catcode`\.\active
            \catcode`\,\active 
            \catcode`\;\active
            \catcode`\:\active
            \catcode`\?\active
            \catcode`\!\active
            \catcode`\/\active 
            \lccode`\~`\~ 	
        }
    \makeatother

    \let\OriginalVerbatim=\Verbatim
    \makeatletter
    \renewcommand{\Verbatim}[1][1]{%
        %\parskip\z@skip
        \sbox\Wrappedcontinuationbox {\Wrappedcontinuationsymbol}%
        \sbox\Wrappedvisiblespacebox {\FV@SetupFont\Wrappedvisiblespace}%
        \def\FancyVerbFormatLine ##1{\hsize\linewidth
            \vtop{\raggedright\hyphenpenalty\z@\exhyphenpenalty\z@
                \doublehyphendemerits\z@\finalhyphendemerits\z@
                \strut ##1\strut}%
        }%
        % If the linebreak is at a space, the latter will be displayed as visible
        % space at end of first line, and a continuation symbol starts next line.
        % Stretch/shrink are however usually zero for typewriter font.
        \def\FV@Space {%
            \nobreak\hskip\z@ plus\fontdimen3\font minus\fontdimen4\font
            \discretionary{\copy\Wrappedvisiblespacebox}{\Wrappedafterbreak}
            {\kern\fontdimen2\font}%
        }%
        
        % Allow breaks at special characters using \PYG... macros.
        \Wrappedbreaksatspecials
        % Breaks at punctuation characters . , ; ? ! and / need catcode=\active 	
        \OriginalVerbatim[#1,codes*=\Wrappedbreaksatpunct]%
    }
    \makeatother

    % Exact colors from NB
    \definecolor{incolor}{HTML}{303F9F}
    \definecolor{outcolor}{HTML}{D84315}
    \definecolor{cellborder}{HTML}{CFCFCF}
    \definecolor{cellbackground}{HTML}{F7F7F7}
    
    % prompt
    \makeatletter
    \newcommand{\boxspacing}{\kern\kvtcb@left@rule\kern\kvtcb@boxsep}
    \makeatother
    \newcommand{\prompt}[4]{
        \ttfamily\llap{{\color{#2}[#3]:\hspace{3pt}#4}}\vspace{-\baselineskip}
    }
    

    
    % Prevent overflowing lines due to hard-to-break entities
    \sloppy 
    % Setup hyperref package
    \hypersetup{
      breaklinks=true,  % so long urls are correctly broken across lines
      colorlinks=true,
      urlcolor=urlcolor,
      linkcolor=linkcolor,
      citecolor=citecolor,
      }
    % Slightly bigger margins than the latex defaults
    
    \geometry{verbose,tmargin=1in,bmargin=1in,lmargin=1in,rmargin=1in}
\begin{document}

\nocite{*} % this command forces all references in template.bib to be printed in the bibliography

\title{Short Rate Models}

\author{
  Ansari, Zain Us Sami Ahmed\\
  \texttt{zainussami@gmail.com}
  \and
  Nguyen, Dang Duy Nghia \\
  \texttt{nghia002@e.ntu.edu.sg}
   \and
  Gunawan, Handy \\
  \texttt{handygunawan17@gmail.com}  
  }

\date{March. 5, 2020} % if this is omitted, the current date is used for the title page

\maketitle

\noindent
\textbf{Keywords:} Short rate models, Interest Rates Derivatives,  Ho-Lee Model, Vasicek Model, Cox-Ingersoll-Ross (CIR) model, Single factor and two factor Hull-White model.



% the following creates an abstract -- it can be omitted
% an example of an environment: these have the form \begin{name} ... \end{name}
\begin{abstract}
In this paper we discuss short rate models with a focus on two factor Hull-White model.  In the first section we discuss models for short rate, in the second section we describe the two factor Hull-White model in mathematical terms, in the third section we compare this model to Ho-Lee Model, Vasicek Model, Cox-Ingersoll-Ross (CIR) model, Single factor Hull-White model,  in the fourth section we present a derivative product payoff using the two factor Hull-White model. and we conclude by discussing the purpose of this paper.  
\end{abstract}

\section{Introduction
}
There are several frameworks to model interest rates, for this paper we will focus on models for short rate.  These models are useful for the valuation of options for which we need an estimate of how much interest rates can change in future.  The short rate $r_t$ is considered to be continually compounded interest rate, where one can borrow or lend money at an infinitesimally short period of time $t$. \\

Short rate models are rooted in the Vasicek model \cite{BN1} which is considered endogenous with time homogeneous parameters (constant over time).  The models extending on the work of Vasicek model have time dependent parameters such as the Hull-White model \cite{HW1}. \\

Short rate models are divided in two categories.  The first one is equilibrium models, Vasicek model and Cox, Ingersoll and Ross (CIR) model are examples of equilibrium short rate models.  The second one is the arbitrage free models, the Ho-Lee model and the Hull-White model are example of arbitrage free models.  We discuss all these models in Section 3. \\ 

The short rate models are either single factor or multi factor, for this paper we will only focus on One-factor short rate models and a two factor short rate model \cite{BA1}.  One factor  models involve a single stochastic factor which is the short rate to determine the future movements of all interest rates.  \\

Single-factor model can accommodate rich patterns however they imply that all rates move in the same direction in any short time interval \cite{cheng2006hull}.


\section{Two-Factor Hull-White Model 
}
\subsection{The motivation for two-factor models
}
The short rate $r_t$ with its stochastic behaviour characterizes the yield curve, which tells the interest rate of bonds in relation to maturity, written as:
\[
P(t,T) = \mathbb{E}\left[e^{-\int_{t}^{T}r(s)ds}\right]
\]
This suggests that if we know the prices of all bonds, we can reconstruct the curve. The quality of the curve, or how well it fits observed real world data, however, depends greatly on how we model the dynamic of $dr_t$. One-factor model like the classic Hull-White does not offer sufficient complexity to fit the curve to actual data, which tends to steepen towards the higher end of the maturity spectrum, but only allow affine transformation to the yield curve. This motivates the introduction to include non-linearity in short rate model to be able to bend the curve more flexibly.

\subsection{The description of Two-Factor Hull-White model
}
One of the way to make the model more flexible is adding a stochastic component $u(t)$ to the $dt$ term of the one-factor Hull-White model, written in full:
\begin{align*}
& dr_t = (\theta(t) + u(t) - \alpha r_t)dt + \sigma_{1}(t)dW_{t}^{1} \text{, where} \\
& u(t) = -\beta u(t)dt + \sigma_{2}(t)dW_{t}^{2}
\end{align*}
such that $dW_{t}^{1}$ and $dW_{t}^{2}$ are two Brownian motions with covariation process $dW_{t}^{1}dW_{t}^{2} = \rho dt$. \\
\\
When the short rate is assumed to evolve in the risk-adjusted measure $\mathbb{P}^{\star}$, we can rewrite the equations as:
\begin{align*}
& dr_t = (\theta(t) + u(t) - \alpha r_t)dt + \sigma_{1}dZ_{t}^{1}  \\
& u(t) = -\beta u(t)dt + \sigma_{2}dZ_{t}^{2}
\end{align*}
where $dZ_{t}^{1}$ and $dZ_{t}^{2}$ are two $\mathbb{P}^{\star}$-Brownian motions and satisfy $dZ_{t}^{1}dZ_{t}^{2} = \bar{\rho} dt$.
$r_{0}, \alpha, \beta, \sigma_{1}, \sigma_{2}$ are positive constants, and $-1 \le \bar{\rho} \le 1$. We define:
\[
\chi (t) = r(t) + \delta u(t)
\]
where $\delta = \frac{1}{\beta - \alpha}$. The differential form of $\chi (t)$ is as follow:
\begin{align*}
d\chi(t) &= (\theta(t) + u(t) - \alpha r(t))dt + \sigma_{1}dZ_{t}^{1}(t) + \delta (-\beta u(t)dt + \sigma_{2}dZ_{t}^{2}) \\
&= (\theta(t) + u(t) - \alpha r(t) - \delta \beta u(t))dt + \sigma_{1}dZ_{t}^{1}(t) + \delta \sigma_{2}dZ_{t}^{2} \\
&= \left(\theta(t) - \alpha r(t) + \left(1 - \frac{\beta}{\beta - \alpha} \right)u(t) \right)dt + \sigma_{3}dZ_{t}^{3} \\
&= (\theta(t) - \alpha \chi(t) )dt + \sigma_{3}dZ_{t}^{3}
\end{align*}
with:
\begin{align*}
\sigma_{3}dZ_{t}^{3} &= \sigma_{1}dZ_{t}^{1} + \delta\sigma_{2}dZ_{t}^{2} \\
(\sigma_{3}dZ_{t}^{3})^{2} &= (\sigma_{1}dZ_{t}^{1} + \delta\sigma_{2}dZ_{t}^{2})^{2} \\
\sigma_{3}^{2}dt &= \sigma_{1}^{2}dt + \delta^{2}\sigma_{2}^{2}dt + 2\delta\sigma_{1}\sigma_{2}dZ_{t}^{1}dZ_{t}^{2} \\
\sigma_{3}^{2}dt  &=  \sigma_{1}^{2}dt + \delta^{2}\sigma_{2}^{2}dt + 2\delta\sigma_{1}\sigma_{2}\bar{\rho} dt \\
\sigma_{3}^{2}  &= \sigma_{1}^{2} + \delta^{2}\sigma_{2}^{2} + 2\delta\sigma_{1}\sigma_{2}\bar{\rho} \\
\sigma_{3}  &= \sqrt{\sigma_{1}^{2} + \delta^{2}\sigma_{2}^{2} + 2\delta\sigma_{1}\sigma_{2}\bar{\rho}} \\
\end{align*}
and 
\[
dZ_{t}^{3} = \frac{\sigma_{1}dZ_{t}^{1} + \delta\sigma_{2}dZ_{t}^{2}}{\sigma_{3}} 
\] \\
\\
Next we define another stochastic process $\psi(t) = -\delta u(t) = \frac{u(t)}{\alpha - \beta}$,
which has the differential form of:
\begin{align*}
d\psi(t) &= -\frac{\beta}{\alpha - \beta}u(t)dt + \frac{\sigma_{2}}{\alpha - \beta}dZ_{t}^{2} \\
&= -\beta\psi(t)dt + \sigma_{4}dZ_{t}^{2}
\end{align*}
where $\sigma_{4} = \frac{\sigma_{2}}{\alpha - \beta}$. Now, we can express $r(t)$ as sum of three independent processes.
\[
r(t) = \tilde{\chi}(t) + \psi(t) + \phi(t)  
\]
where
\begin{align*}
&d\tilde{\chi}(t) = -\alpha\tilde{\chi}(t)dt + \sigma_{3}dZ_{t}^{3} \\
&d\psi(t) = -\beta\psi(t)dt + \sigma_{4}dZ_{t}^{2} \\
&\phi(t) = r_{0}e^{-\alpha t} + \int_{0}^{t}{\theta(s)e^{-\alpha(t-s)}ds} \\
\end{align*}
In this form, we can see that the Two-Factor Hull-White model is equivalent to a Two-Addictive-Factor Gaussian model.
However, this is beyond the scope of this paper, see \cite{BA1}.
\section{Comparison with Single Factor Short Rate Models
}

\subsection{Comparison with Vasicek model
}
The Vasicek Model \cite{Vas1} model is one of the earliest stochastic models of the term structure of interest rate.\\

The Vasicek model describes short interest rates as follows,
 \[
dr_t = \alpha(\beta - r_t) dt + \sigma dW_t
\]
Where,\\
$\alpha =$ Speed of mean reversion $0 \leq \alpha \leq 1 $ \\
$\beta =$ Mean reversion level  \\
$r_t$ = Short rate  \\
$\sigma$ = Short rate volatility \\
$W_t$ = Brownian motion process  \\

The model exhibits mean reversion, over time interest rate converge to the mean reversion level $\beta$ at the speed of $\alpha$.  In comparison to the two factor Hull-White model it exhibits several short comings such as there is no term structure volatility, there is only one factor which assumes short rates correlated, it is an endogenous model containing time homogeneous parameters which do not vary over time.


\subsection{Comparison with Cox-Ingersoll-Ross (CIR) model
}

The Cox-Ingersoll-Ross (CIR) model was developed in 1985 \cite{CIR1} by J. C. Cox, J. E. Ingersoll and S. A. Ross.  This model is considered an improvement to the Vasicek model and the square root element does not allow for negative rates.\\

The Cox-Ingersoll-Ross (CIR) model describes short interest rates as follows,
 \[
dr_t = \alpha(\beta - r_t) dt + \sigma \sqrt{r_t} dW_t
\]
Where,\\
$\alpha =$ Speed of mean reversion $0 \leq \alpha \leq 1 $ \\
$\beta =$ Mean reversion level  \\
$r_t$ = Short rate  \\
$\sigma$ = Short rate volatility \\
$W_t$ = Brownian motion process  \\

The model exhibits mean reversion, over time interest rate converge to the mean reversion level $\beta$ at the speed of $\alpha$.  In comparison to the two factor Hull-White model it exhibits several short comings such as there is no term structure volatility, there is only one factor which assumes short rates correlated, it is an endogenous model containing time homogeneous parameters which do not vary over time.

\subsection{Comparison with Ho-Lee model
}
Ho-Lee model developed in 1986 \cite{HoLee1} by Thomas SY Ho and Sang-Bin Lee, it is a single factor short rate model used in pricing of interest rate derivatives.  It was the first model arbitrage free computational lattices for evolution of short interest rates.  Ho-Lee assumed interest rates have a normal distribution. \\

The Ho-Lee  model describes short interest rates as follows,
 \[
dr_t = \theta_t dt + \sigma  dW_t
\]
Where,\\
$\theta_t =$ Drift  \\
$\sigma$ = Short rate volatility \\
$W_t$ = Brownian motion process  \\

This model permits both the drift to be function of time and it produces an instantaneous short interest rate that has a normal distribution it exhibits no mean reversion.  The Ho-Lee model is tractable because it is Markov and provides and exact fit in the current term structure, however it gives very little flexibility in choosing the volatility structure.  Since it has no mean reversion no matter how high or low the interest rates are the average direction of interest rates movement over the next short period of time is same.

\subsection{Comparison with Hull-White model
}
Hull-White model developed in 1990 \cite{HW1} by John C. Hull and Alan D. White, it is a single factor short rate model used in pricing of interest rate derivatives.  It is also known as the Extended Vasicek Model.\\

The Hull-White model in essence is the same as the Vasicek model with time dependent parameters and describes short interest rates as follows,
 \[
dr_t = (\theta_t - \alpha_t r_t) dt + \sigma_t  dW_t
\]
Where,\\
$\theta_t =$ Drift  \\
$\alpha_t =$ Time dependent Speed of mean reversion $0 \leq \alpha \leq 1 $ \\
$\sigma_t$ = Time dependent Short rate volatility \\
$W_t$ = Brownian motion process  \\

The Hull-White model also assumes that short rates have a normal distribution it exhibits mean reversion.  This model provides same amount of traceability as the Ho-Lee model.  The model can represent a wider range of volatility structures compared to the Ho-Lee model.
\section{Derivative product Payoff}
In this section, we will provide a numerical example of calibration of
2-factor Hull-White model. We will fit the model to historical data of
swaptions volatility and obtain the five parameters
\(\alpha, \beta, \sigma_{1}, \sigma_{2}, \rho\)

    \begin{tcolorbox}[breakable, size=fbox, boxrule=1pt, pad at break*=1mm,colback=cellbackground, colframe=cellborder]
\prompt{In}{incolor}{1}{\boxspacing}
\begin{Verbatim}[commandchars=\\\{\}]
\PY{k+kn}{import} \PY{n+nn}{QuantLib} \PY{k}{as} \PY{n+nn}{ql}
\PY{k+kn}{from} \PY{n+nn}{collections} \PY{k+kn}{import} \PY{n}{namedtuple}
\PY{k+kn}{import} \PY{n+nn}{math}
\end{Verbatim}
\end{tcolorbox}

    \begin{tcolorbox}[breakable, size=fbox, boxrule=1pt, pad at break*=1mm,colback=cellbackground, colframe=cellborder]
\prompt{In}{incolor}{2}{\boxspacing}
\begin{Verbatim}[commandchars=\\\{\}]
\PY{n}{today} \PY{o}{=} \PY{n}{ql}\PY{o}{.}\PY{n}{Date}\PY{p}{(}\PY{l+m+mi}{15}\PY{p}{,} \PY{n}{ql}\PY{o}{.}\PY{n}{February}\PY{p}{,} \PY{l+m+mi}{2020}\PY{p}{)}\PY{p}{;}
\PY{n}{settlement}\PY{o}{=} \PY{n}{ql}\PY{o}{.}\PY{n}{Date}\PY{p}{(}\PY{l+m+mi}{19}\PY{p}{,} \PY{n}{ql}\PY{o}{.}\PY{n}{February}\PY{p}{,} \PY{l+m+mi}{2020}\PY{p}{)}\PY{p}{;}
\PY{n}{ql}\PY{o}{.}\PY{n}{Settings}\PY{o}{.}\PY{n}{instance}\PY{p}{(}\PY{p}{)}\PY{o}{.}\PY{n}{evaluationDate} \PY{o}{=} \PY{n}{today}\PY{p}{;}
\PY{n}{term\PYZus{}structure} \PY{o}{=} \PY{n}{ql}\PY{o}{.}\PY{n}{YieldTermStructureHandle}\PY{p}{(}
    \PY{n}{ql}\PY{o}{.}\PY{n}{FlatForward}\PY{p}{(}\PY{n}{settlement}\PY{p}{,}\PY{l+m+mf}{0.04875825}\PY{p}{,}\PY{n}{ql}\PY{o}{.}\PY{n}{Actual365Fixed}\PY{p}{(}\PY{p}{)}\PY{p}{)}
    \PY{p}{)}
\PY{n}{index} \PY{o}{=} \PY{n}{ql}\PY{o}{.}\PY{n}{Euribor1Y}\PY{p}{(}\PY{n}{term\PYZus{}structure}\PY{p}{)}
\end{Verbatim}
\end{tcolorbox}

    We are going to calibrate to the swaption volatilities as shown below.

    \begin{tcolorbox}[breakable, size=fbox, boxrule=1pt, pad at break*=1mm,colback=cellbackground, colframe=cellborder]
\prompt{In}{incolor}{3}{\boxspacing}
\begin{Verbatim}[commandchars=\\\{\}]
\PY{n}{CalibrationData} \PY{o}{=} \PY{n}{namedtuple}\PY{p}{(}\PY{l+s+s2}{\PYZdq{}}\PY{l+s+s2}{CalibrationData}\PY{l+s+s2}{\PYZdq{}}\PY{p}{,} 
                             \PY{l+s+s2}{\PYZdq{}}\PY{l+s+s2}{start, length, volatility}\PY{l+s+s2}{\PYZdq{}}\PY{p}{)}
\PY{n}{data} \PY{o}{=} \PY{p}{[}\PY{n}{CalibrationData}\PY{p}{(}\PY{l+m+mi}{1}\PY{p}{,} \PY{l+m+mi}{5}\PY{p}{,} \PY{l+m+mf}{0.1148}\PY{p}{)}\PY{p}{,}
        \PY{n}{CalibrationData}\PY{p}{(}\PY{l+m+mi}{2}\PY{p}{,} \PY{l+m+mi}{4}\PY{p}{,} \PY{l+m+mf}{0.1108}\PY{p}{)}\PY{p}{,}
        \PY{n}{CalibrationData}\PY{p}{(}\PY{l+m+mi}{3}\PY{p}{,} \PY{l+m+mi}{3}\PY{p}{,} \PY{l+m+mf}{0.1070}\PY{p}{)}\PY{p}{,}
        \PY{n}{CalibrationData}\PY{p}{(}\PY{l+m+mi}{4}\PY{p}{,} \PY{l+m+mi}{2}\PY{p}{,} \PY{l+m+mf}{0.1021}\PY{p}{)}\PY{p}{,}
        \PY{n}{CalibrationData}\PY{p}{(}\PY{l+m+mi}{5}\PY{p}{,} \PY{l+m+mi}{1}\PY{p}{,} \PY{l+m+mf}{0.1000} \PY{p}{)}\PY{p}{]}
\end{Verbatim}
\end{tcolorbox}

    \begin{tcolorbox}[breakable, size=fbox, boxrule=1pt, pad at break*=1mm,colback=cellbackground, colframe=cellborder]
\prompt{In}{incolor}{4}{\boxspacing}
\begin{Verbatim}[commandchars=\\\{\}]
\PY{c+c1}{\PYZsh{} \PYZlt{}!\PYZhy{}\PYZhy{} collapse=True \PYZhy{}\PYZhy{}\PYZgt{}}
\PY{k}{def} \PY{n+nf}{create\PYZus{}swaption\PYZus{}helpers}\PY{p}{(}\PY{n}{data}\PY{p}{,} \PY{n}{index}\PY{p}{,} \PY{n}{term\PYZus{}structure}\PY{p}{,} \PY{n}{engine}\PY{p}{)}\PY{p}{:}
    \PY{n}{swaptions} \PY{o}{=} \PY{p}{[}\PY{p}{]}
    \PY{n}{fixed\PYZus{}leg\PYZus{}tenor} \PY{o}{=} \PY{n}{ql}\PY{o}{.}\PY{n}{Period}\PY{p}{(}\PY{l+m+mi}{1}\PY{p}{,} \PY{n}{ql}\PY{o}{.}\PY{n}{Years}\PY{p}{)}
    \PY{n}{fixed\PYZus{}leg\PYZus{}daycounter} \PY{o}{=} \PY{n}{ql}\PY{o}{.}\PY{n}{Actual360}\PY{p}{(}\PY{p}{)}
    \PY{n}{floating\PYZus{}leg\PYZus{}daycounter} \PY{o}{=} \PY{n}{ql}\PY{o}{.}\PY{n}{Actual360}\PY{p}{(}\PY{p}{)}
    \PY{k}{for} \PY{n}{d} \PY{o+ow}{in} \PY{n}{data}\PY{p}{:}
        \PY{n}{vol\PYZus{}handle} \PY{o}{=} \PY{n}{ql}\PY{o}{.}\PY{n}{QuoteHandle}\PY{p}{(}\PY{n}{ql}\PY{o}{.}\PY{n}{SimpleQuote}\PY{p}{(}\PY{n}{d}\PY{o}{.}\PY{n}{volatility}\PY{p}{)}\PY{p}{)}
        \PY{n}{helper} \PY{o}{=} \PY{n}{ql}\PY{o}{.}\PY{n}{SwaptionHelper}\PY{p}{(}\PY{n}{ql}\PY{o}{.}\PY{n}{Period}\PY{p}{(}\PY{n}{d}\PY{o}{.}\PY{n}{start}\PY{p}{,} \PY{n}{ql}\PY{o}{.}\PY{n}{Years}\PY{p}{)}\PY{p}{,}
                                   \PY{n}{ql}\PY{o}{.}\PY{n}{Period}\PY{p}{(}\PY{n}{d}\PY{o}{.}\PY{n}{length}\PY{p}{,} \PY{n}{ql}\PY{o}{.}\PY{n}{Years}\PY{p}{)}\PY{p}{,}
                                   \PY{n}{vol\PYZus{}handle}\PY{p}{,}
                                   \PY{n}{index}\PY{p}{,}
                                   \PY{n}{fixed\PYZus{}leg\PYZus{}tenor}\PY{p}{,}
                                   \PY{n}{fixed\PYZus{}leg\PYZus{}daycounter}\PY{p}{,}
                                   \PY{n}{floating\PYZus{}leg\PYZus{}daycounter}\PY{p}{,}
                                   \PY{n}{term\PYZus{}structure}
                                   \PY{p}{)}
        \PY{n}{helper}\PY{o}{.}\PY{n}{setPricingEngine}\PY{p}{(}\PY{n}{engine}\PY{p}{)}
        \PY{n}{swaptions}\PY{o}{.}\PY{n}{append}\PY{p}{(}\PY{n}{helper}\PY{p}{)}
    \PY{k}{return} \PY{n}{swaptions}    

\PY{k}{def} \PY{n+nf}{calibration\PYZus{}report}\PY{p}{(}\PY{n}{swaptions}\PY{p}{,} \PY{n}{data}\PY{p}{)}\PY{p}{:}
    \PY{n+nb}{print}\PY{p}{(}\PY{l+s+s2}{\PYZdq{}}\PY{l+s+s2}{\PYZhy{}}\PY{l+s+s2}{\PYZdq{}}\PY{o}{*}\PY{l+m+mi}{82}\PY{p}{)}
    \PY{n+nb}{print}\PY{p}{(}\PY{l+s+s2}{\PYZdq{}}\PY{l+s+si}{\PYZpc{}15s}\PY{l+s+s2}{ }\PY{l+s+si}{\PYZpc{}15s}\PY{l+s+s2}{ }\PY{l+s+si}{\PYZpc{}15s}\PY{l+s+s2}{ }\PY{l+s+si}{\PYZpc{}15s}\PY{l+s+s2}{ }\PY{l+s+si}{\PYZpc{}15s}\PY{l+s+s2}{\PYZdq{}} \PY{o}{\PYZpc{}} \PYZbs{}
    \PY{p}{(}\PY{l+s+s2}{\PYZdq{}}\PY{l+s+s2}{Model Price}\PY{l+s+s2}{\PYZdq{}}\PY{p}{,} \PY{l+s+s2}{\PYZdq{}}\PY{l+s+s2}{Market Price}\PY{l+s+s2}{\PYZdq{}}\PY{p}{,} \PY{l+s+s2}{\PYZdq{}}\PY{l+s+s2}{Implied Vol}\PY{l+s+s2}{\PYZdq{}}\PY{p}{,} \PY{l+s+s2}{\PYZdq{}}\PY{l+s+s2}{Market Vol}\PY{l+s+s2}{\PYZdq{}}\PY{p}{,} \PY{l+s+s2}{\PYZdq{}}\PY{l+s+s2}{Rel Error}\PY{l+s+s2}{\PYZdq{}}\PY{p}{)}\PY{p}{)}
    \PY{n+nb}{print}\PY{p}{(}\PY{l+s+s2}{\PYZdq{}}\PY{l+s+s2}{\PYZhy{}}\PY{l+s+s2}{\PYZdq{}}\PY{o}{*}\PY{l+m+mi}{82}\PY{p}{)}
    \PY{n}{cum\PYZus{}err} \PY{o}{=} \PY{l+m+mf}{0.0}
    \PY{k}{for} \PY{n}{i}\PY{p}{,} \PY{n}{s} \PY{o+ow}{in} \PY{n+nb}{enumerate}\PY{p}{(}\PY{n}{swaptions}\PY{p}{)}\PY{p}{:}
        \PY{n}{model\PYZus{}price} \PY{o}{=} \PY{n}{s}\PY{o}{.}\PY{n}{modelValue}\PY{p}{(}\PY{p}{)}
        \PY{n}{market\PYZus{}vol} \PY{o}{=} \PY{n}{data}\PY{p}{[}\PY{n}{i}\PY{p}{]}\PY{o}{.}\PY{n}{volatility}
        \PY{n}{black\PYZus{}price} \PY{o}{=} \PY{n}{s}\PY{o}{.}\PY{n}{blackPrice}\PY{p}{(}\PY{n}{market\PYZus{}vol}\PY{p}{)}
        \PY{n}{rel\PYZus{}error} \PY{o}{=} \PY{n}{model\PYZus{}price}\PY{o}{/}\PY{n}{black\PYZus{}price} \PY{o}{\PYZhy{}} \PY{l+m+mf}{1.0}
        \PY{n}{implied\PYZus{}vol} \PY{o}{=} \PY{n}{s}\PY{o}{.}\PY{n}{impliedVolatility}\PY{p}{(}\PY{n}{model\PYZus{}price}\PY{p}{,}
                                          \PY{l+m+mf}{1e\PYZhy{}5}\PY{p}{,} \PY{l+m+mi}{50}\PY{p}{,} \PY{l+m+mf}{0.0}\PY{p}{,} \PY{l+m+mf}{0.50}\PY{p}{)}
        \PY{n}{rel\PYZus{}error2} \PY{o}{=} \PY{n}{implied\PYZus{}vol}\PY{o}{/}\PY{n}{market\PYZus{}vol}\PY{o}{\PYZhy{}}\PY{l+m+mf}{1.0}
        \PY{n}{cum\PYZus{}err} \PY{o}{+}\PY{o}{=} \PY{n}{rel\PYZus{}error2}\PY{o}{*}\PY{n}{rel\PYZus{}error2}
        
        \PY{n+nb}{print}\PY{p}{(}\PY{l+s+s2}{\PYZdq{}}\PY{l+s+si}{\PYZpc{}15.5f}\PY{l+s+s2}{ }\PY{l+s+si}{\PYZpc{}15.5f}\PY{l+s+s2}{ }\PY{l+s+si}{\PYZpc{}15.5f}\PY{l+s+s2}{ }\PY{l+s+si}{\PYZpc{}15.5f}\PY{l+s+s2}{ }\PY{l+s+si}{\PYZpc{}15.5f}\PY{l+s+s2}{\PYZdq{}} \PY{o}{\PYZpc{}} \PYZbs{}
        \PY{p}{(}\PY{n}{model\PYZus{}price}\PY{p}{,} \PY{n}{black\PYZus{}price}\PY{p}{,} \PY{n}{implied\PYZus{}vol}\PY{p}{,} \PY{n}{market\PYZus{}vol}\PY{p}{,} \PY{n}{rel\PYZus{}error}\PY{p}{)}\PY{p}{)}
    \PY{n+nb}{print}\PY{p}{(}\PY{l+s+s2}{\PYZdq{}}\PY{l+s+s2}{\PYZhy{}}\PY{l+s+s2}{\PYZdq{}}\PY{o}{*}\PY{l+m+mi}{82}\PY{p}{)}
    \PY{n+nb}{print}\PY{p}{(}\PY{l+s+s2}{\PYZdq{}}\PY{l+s+s2}{Cumulative Error : }\PY{l+s+si}{\PYZpc{}15.5f}\PY{l+s+s2}{\PYZdq{}} \PY{o}{\PYZpc{}} \PY{n}{math}\PY{o}{.}\PY{n}{sqrt}\PY{p}{(}\PY{n}{cum\PYZus{}err}\PY{p}{)}\PY{p}{)}
\end{Verbatim}
\end{tcolorbox}

    \hypertarget{g2-model}{%
\subsection{G2++ Model}\label{g2-model}}

By \cite{BA1} we know that G2++ model is equivalent to Two-Factor
Hull-White model. Since Quantlib only support G2++ at the moment, we
will calibrate G2++ model first to get its 5 parameters. After that we
can map them uniquely to the set of Hull-White parameters.

G2++ model is given by: \begin{equation}
dr_t = \varphi(t) + x_t + y_t
\end{equation}

where \$ x\_t \$ and \$ y\_t \$ are defined by

\begin{align*}
&dx_t = -a x_t dt + \sigma dW^1_t\nonumber \\
&dy_t = -b y_t dt + \eta dW^2_t \nonumber \\
&\left<dW^1_t dW^2_t\right> = \rho dt 
\end{align*}

We use the \texttt{TreeSwaptionEngine} to value the swaptions in the
calibration step.

    \begin{tcolorbox}[breakable, size=fbox, boxrule=1pt, pad at break*=1mm,colback=cellbackground, colframe=cellborder]
\prompt{In}{incolor}{5}{\boxspacing}
\begin{Verbatim}[commandchars=\\\{\}]
\PY{n}{model} \PY{o}{=} \PY{n}{ql}\PY{o}{.}\PY{n}{G2}\PY{p}{(}\PY{n}{term\PYZus{}structure}\PY{p}{)}\PY{p}{;}
\PY{n}{engine} \PY{o}{=} \PY{n}{ql}\PY{o}{.}\PY{n}{TreeSwaptionEngine}\PY{p}{(}\PY{n}{model}\PY{p}{,} \PY{l+m+mi}{25}\PY{p}{)}
\PY{n}{swaptions} \PY{o}{=} \PY{n}{create\PYZus{}swaption\PYZus{}helpers}\PY{p}{(}\PY{n}{data}\PY{p}{,} \PY{n}{index}\PY{p}{,} \PY{n}{term\PYZus{}structure}\PY{p}{,} \PY{n}{engine}\PY{p}{)}
\end{Verbatim}
\end{tcolorbox}

    \begin{tcolorbox}[breakable, size=fbox, boxrule=1pt, pad at break*=1mm,colback=cellbackground, colframe=cellborder]
\prompt{In}{incolor}{6}{\boxspacing}
\begin{Verbatim}[commandchars=\\\{\}]
\PY{n}{optimization\PYZus{}method} \PY{o}{=} \PY{n}{ql}\PY{o}{.}\PY{n}{LevenbergMarquardt}\PY{p}{(}\PY{l+m+mf}{1.0e\PYZhy{}8}\PY{p}{,}\PY{l+m+mf}{1.0e\PYZhy{}8}\PY{p}{,}\PY{l+m+mf}{1.0e\PYZhy{}8}\PY{p}{)}
\PY{n}{end\PYZus{}criteria} \PY{o}{=} \PY{n}{ql}\PY{o}{.}\PY{n}{EndCriteria}\PY{p}{(}\PY{l+m+mi}{1000}\PY{p}{,} \PY{l+m+mi}{100}\PY{p}{,} \PY{l+m+mf}{1e\PYZhy{}6}\PY{p}{,} \PY{l+m+mf}{1e\PYZhy{}8}\PY{p}{,} \PY{l+m+mf}{1e\PYZhy{}8}\PY{p}{)}
\PY{n}{model}\PY{o}{.}\PY{n}{calibrate}\PY{p}{(}\PY{n}{swaptions}\PY{p}{,} \PY{n}{optimization\PYZus{}method}\PY{p}{,} \PY{n}{end\PYZus{}criteria}\PY{p}{)}

\PY{n}{a}\PY{p}{,} \PY{n}{sigma}\PY{p}{,} \PY{n}{b}\PY{p}{,} \PY{n}{eta}\PY{p}{,} \PY{n}{rho} \PY{o}{=} \PY{n}{model}\PY{o}{.}\PY{n}{params}\PY{p}{(}\PY{p}{)}
\PY{n+nb}{print}\PY{p}{(}\PY{l+s+s2}{\PYZdq{}}\PY{l+s+s2}{G2++ parameters after calibration are:}\PY{l+s+se}{\PYZbs{}n}\PY{l+s+s2}{a = }\PY{l+s+si}{\PYZpc{}6.5f}\PY{l+s+s2}{, sigma = }\PY{l+s+si}{\PYZpc{}6.5f}\PY{l+s+s2}{, b = }\PY{l+s+si}{\PYZpc{}6.5f}\PY{l+s+s2}{, eta = }\PY{l+s+si}{\PYZpc{}6.5f}\PY{l+s+s2}{, rho = }\PY{l+s+si}{\PYZpc{}6.5f}\PY{l+s+s2}{ }\PY{l+s+s2}{\PYZdq{}} \PY{o}{\PYZpc{}} \PY{p}{(}\PY{n}{a}\PY{p}{,} \PY{n}{sigma}\PY{p}{,} \PY{n}{b}\PY{p}{,} \PY{n}{eta}\PY{p}{,} \PY{n}{rho}\PY{p}{)}\PY{p}{)}
\end{Verbatim}
\end{tcolorbox}

    \begin{Verbatim}[commandchars=\\\{\}]
G2++ parameters after calibration are:
a = 0.04810, sigma = 0.00301, b = 0.03892, eta = 0.00472, rho = 0.03977
    \end{Verbatim}

    \hypertarget{mapping-from-g2-to-2-factor-hull-white-parameters}{%
\subsection{Mapping from G2++ to 2-factor Hull-White
parameters}\label{mapping-from-g2-to-2-factor-hull-white-parameters}}

By \cite{BA1} the mapping between G2++ model and 2-factor Hull-White
model are as follow: \begin{align*}
\alpha &= a \\
\beta &= b \\
\sigma_{1} &= \sqrt{\sigma^{2} + \eta^{2} + 2\rho\sigma\eta} \\
\sigma_{2} &= \eta(a - b) \\
\bar{\rho} &= \frac{\sigma\rho + \eta}{\sigma_{1}} \\
\end{align*}

    \begin{tcolorbox}[breakable, size=fbox, boxrule=1pt, pad at break*=1mm,colback=cellbackground, colframe=cellborder]
\prompt{In}{incolor}{7}{\boxspacing}
\begin{Verbatim}[commandchars=\\\{\}]
\PY{k}{def} \PY{n+nf}{to\PYZus{}2FWH\PYZus{}params}\PY{p}{(}\PY{n}{a}\PY{p}{,} \PY{n}{sigma}\PY{p}{,} \PY{n}{b}\PY{p}{,} \PY{n}{eta}\PY{p}{,} \PY{n}{rho}\PY{p}{)}\PY{p}{:}
    \PY{n}{alpha} \PY{o}{=} \PY{n}{a}
    \PY{n}{beta} \PY{o}{=} \PY{n}{b}
    \PY{n}{sigma\PYZus{}1} \PY{o}{=} \PY{n}{math}\PY{o}{.}\PY{n}{sqrt}\PY{p}{(}\PY{n}{sigma}\PY{o}{*}\PY{o}{*}\PY{l+m+mi}{2} \PY{o}{+} \PY{n}{eta}\PY{o}{*}\PY{o}{*}\PY{l+m+mi}{2} \PY{o}{+} \PY{l+m+mi}{2}\PY{o}{*}\PY{n}{rho}\PY{o}{*}\PY{n}{sigma}\PY{o}{*}\PY{n}{eta}\PY{p}{)}
    \PY{n}{sigma\PYZus{}2} \PY{o}{=} \PY{n}{eta}\PY{o}{*}\PY{p}{(}\PY{n}{a} \PY{o}{\PYZhy{}} \PY{n}{b}\PY{p}{)}
    \PY{n}{rho\PYZus{}bar} \PY{o}{=} \PY{p}{(}\PY{n}{sigma}\PY{o}{*}\PY{n}{rho} \PY{o}{+} \PY{n}{eta}\PY{p}{)}\PY{o}{/}\PY{n}{sigma\PYZus{}1}
    \PY{k}{return} \PY{n}{alpha}\PY{p}{,} \PY{n}{beta}\PY{p}{,} \PY{n}{sigma\PYZus{}1}\PY{p}{,} \PY{n}{sigma\PYZus{}2}\PY{p}{,} \PY{n}{rho\PYZus{}bar}
\PY{n}{test} \PY{o}{=} \PY{n}{to\PYZus{}2FWH\PYZus{}params}\PY{p}{(}\PY{l+m+mf}{0.521159}\PY{p}{,} \PY{l+m+mf}{0.005779}\PY{p}{,} \PY{l+m+mf}{0.075631}\PY{p}{,} \PY{l+m+mf}{0.011573}\PY{p}{,} \PY{o}{\PYZhy{}}\PY{l+m+mf}{0.986876}\PY{p}{)}
\PY{k}{assert}\PY{p}{(}\PY{n}{test} \PY{o}{==} \PY{p}{(}\PY{l+m+mf}{0.521159}\PY{p}{,} \PY{l+m+mf}{0.075631}\PY{p}{,} \PY{l+m+mf}{0.005943560538348711}\PY{p}{,} \PY{l+m+mf}{0.005156095544000001}\PY{p}{,} \PY{l+m+mf}{0.9875971748124582}\PY{p}{)}\PY{p}{)}
\end{Verbatim}
\end{tcolorbox}

    \begin{tcolorbox}[breakable, size=fbox, boxrule=1pt, pad at break*=1mm,colback=cellbackground, colframe=cellborder]
\prompt{In}{incolor}{8}{\boxspacing}
\begin{Verbatim}[commandchars=\\\{\}]
\PY{n}{alpha}\PY{p}{,} \PY{n}{beta}\PY{p}{,} \PY{n}{sigma\PYZus{}1}\PY{p}{,} \PY{n}{sigma\PYZus{}2}\PY{p}{,} \PY{n}{rho\PYZus{}bar} \PY{o}{=} \PY{n}{to\PYZus{}2FWH\PYZus{}params}\PY{p}{(}\PY{n}{a}\PY{p}{,} \PY{n}{sigma}\PY{p}{,} \PY{n}{b}\PY{p}{,} \PY{n}{eta}\PY{p}{,} \PY{n}{rho}\PY{p}{)}
\PY{n+nb}{print}\PY{p}{(}\PY{l+s+s2}{\PYZdq{}}\PY{l+s+s2}{Two\PYZhy{}factor Hull\PYZhy{}White parameters after calibration are:}\PY{l+s+se}{\PYZbs{}n}\PY{l+s+s2}{alpha = }\PY{l+s+si}{\PYZpc{}6.5f}\PY{l+s+s2}{, beta = }\PY{l+s+si}{\PYZpc{}6.5f}\PY{l+s+s2}{, sigma\PYZus{}1 = }\PY{l+s+si}{\PYZpc{}6.5f}\PY{l+s+s2}{, sigma\PYZus{}2 = }\PY{l+s+si}{\PYZpc{}6.5f}\PY{l+s+s2}{, rho\PYZus{}bar = }\PY{l+s+si}{\PYZpc{}6.5f}\PY{l+s+s2}{ }\PY{l+s+s2}{\PYZdq{}} \PY{o}{\PYZpc{}} \PY{p}{(}\PY{n}{alpha}\PY{p}{,} \PY{n}{beta}\PY{p}{,} \PY{n}{sigma\PYZus{}1}\PY{p}{,} \PY{n}{sigma\PYZus{}2}\PY{p}{,} \PY{n}{rho\PYZus{}bar}\PY{p}{)}\PY{p}{)}
\end{Verbatim}
\end{tcolorbox}

    \begin{Verbatim}[commandchars=\\\{\}]
Two-factor Hull-White parameters after calibration are:
alpha = 0.04810, beta = 0.03892, sigma\_1 = 0.00570, sigma\_2 = 0.00004, rho\_bar =
0.84939
    \end{Verbatim}

    \begin{tcolorbox}[breakable, size=fbox, boxrule=1pt, pad at break*=1mm,colback=cellbackground, colframe=cellborder]
\prompt{In}{incolor}{9}{\boxspacing}
\begin{Verbatim}[commandchars=\\\{\}]
\PY{n}{calibration\PYZus{}report}\PY{p}{(}\PY{n}{swaptions}\PY{p}{,} \PY{n}{data}\PY{p}{)}
\end{Verbatim}
\end{tcolorbox}

    \begin{Verbatim}[commandchars=\\\{\}]
--------------------------------------------------------------------------------
--
    Model Price    Market Price     Implied Vol      Market Vol       Rel Error
--------------------------------------------------------------------------------
--
        0.00871         0.00949         0.10531         0.11480        -0.08263
        0.00968         0.01008         0.10634         0.11080        -0.04018
        0.00867         0.00871         0.10652         0.10700        -0.00448
        0.00653         0.00625         0.10665         0.10210         0.04442
        0.00357         0.00334         0.10680         0.10000         0.06773
--------------------------------------------------------------------------------
--
Cumulative Error :         0.12288
    \end{Verbatim}

\section{Conclusion}

In this paper we went through several examples of single factor short rate models and also discussed the two factor Hull-White model in detail.   The purpose of this paper was to compare and contrast the Two-Factor Hull-White Model with other interest short rate models and present a derivative product payoff,  a complete numerical example using the two-Factor Hull-White model.  However in this paper we do not ask question like "what number of factor to use?" .

\newpage
\section*{} \label{bibsection}


% the second parameter MMMMM should be as long as the longest label you use, in my case Smoller -- if you use % numbers only, use 99
% use \cite{refname} to refer to bibliography item \bibitem{refname} 
% LaTeX assigns a number, unless you use \bibitem[Name]{refname} -- in this case
% LaTeX prints Name when you use \cite{refname}
\begin{thebibliography}{MMMMM} 
\bibitem{BA1} Blanchard, Arnaud. "The two-factor Hull-White model: pricing and calibration of interest rates derivatives." KTH Royal Insitute of Technology (2014).
\bibitem{HoLee1} Ho, Thomas SY, and Sang‐Bin Lee. "Term structure movements and pricing interest rate contingent claims." the Journal of Finance 41.5 (1986): 1011-1029.
\bibitem{Vas1} Vasicek, Oldrich. "An equilibrium characterization of the term structure." Journal of financial economics 5.2 (1977): 177-188.
\bibitem{CIR1} Cox, John C., Jonathan E. Ingersoll Jr, and Stephen A. Ross. "An intertemporal general equilibrium model of asset prices." Econometrica: Journal of the Econometric Society (1985): 363-384.
\bibitem{HW1} Hull, John, and Alan White. "Pricing interest-rate-derivative securities." The review of financial studies 3.4 (1990): 573-592.
\bibitem{HW2} Hull, John, and Alan White. "One-Factor Interest Rate Models and the Valuation of Interest Rate Derivative Securities" Journal of Finance and Quantitative Analysis, volume 28,  (1993): 235 - 254.
\bibitem{HW3} Hull, John, and Alan White. "Numerical procedures for implementing term structure models I: Single-factor models." Journal of derivatives 2.1 (1994): 7-16.
\bibitem{HW4} Hull, John, and Alan White. "The general Hull–White model and supercalibration." Financial Analysts Journal 57.6 (2001): 34-43.
\bibitem{HW5} Hull, John. "Using Hull-White interest rate trees." Journal of derivatives 3.3 (1996): 26-36.
\bibitem{BN1} Burgess, Nicholas, An Overview of the Vasicek Short Rate Model (August 12, 2014). Available at SSRN: https://ssrn.com/abstract=2479671 or http://dx.doi.org/10.2139/ssrn.2479671
\bibitem{HJM1} Heath, David, Robert Jarrow, and Andrew Morton. "Bond pricing and the term structure of interest rates: A new methodology for contingent claims valuation." Econometrica: Journal of the Econometric Society (1992): 77-105.
\bibitem{cheng2006hull} Cheng, Chia-Jen "On Hull-White Models: One and Two Factors (2006).
\bibitem{Jam1} Jamshidian, Farshid. "An exact bond option formula." The journal of Finance 44.1 (1989): 205-209.
\bibitem{MT1}  Turnbull, Stuart M., and Frank Milne. "A simple approach to interest-rate option pricing." The Review of Financial Studies 4.1 (1991): 87-120.

\end{thebibliography}
\bibliographystyle{plain}
\bibliography{template}

\end{document}
