\documentclass{article}
\usepackage{graphicx}
% stuff from the percent sign to end of line is a comment, ignored by LaTeX

\usepackage{amsmath,amssymb,graphicx} %load extra symbols and environments
\usepackage[margin=1in]{geometry} %set margins
\usepackage{enumerate}
\begin{document}

\nocite{*} % this command forces all references in template.bib to be printed in the bibliography

\title{Short Rate Models}

\author{
  Ansari, Zain Us Sami Ahmed\\
  \texttt{zainussami@gmail.com}
  \and
  Nguyen, Dang Duy Nghia \\
  \texttt{nghia002@e.ntu.edu.sg}
   \and
  Gunawan, Handy \\
  \texttt{handygunawan17@gmail.com}  
  }

\date{March. 5, 2020} % if this is omitted, the current date is used for the title page

\maketitle

\noindent
\textbf{Keywords:} Short rate models, Interest Rates Derivatives,  Ho-Lee Model, Vasicek Model, Cox-Ingersoll-Ross (CIR) model, Single factor and two factor Hull-White model.



% the following creates an abstract -- it can be omitted
% an example of an environment: these have the form \begin{name} ... \end{name}
\begin{abstract}
In this paper we discuss short rate models with a focus on two factor Hull-White model.  In the first section we discuss models for short rate, in the second section we describe the two factor Hull-White model in mathematical terms, in the third section we compare this model to Ho-Lee Model, Vasicek Model, Cox-Ingersoll-Ross (CIR) model, Single factor Hull-White model,  in the fourth section we present a derivative product payoff using the two factor Hull-White model. and we conclude by discussing the purpose of this paper.  
\end{abstract}

\section{Introduction
}
There are several frameworks to model interest rates, for this paper we will focus on models for short rate.  These models are useful for the valuation of options for which we need an estimate of how much interest rates can change in future.  The short rate $r_t$ is considered to be continually compounded interest rate, where one can borrow or lend money at an infinitesimally short period of time $t$. \\

Short rate models are rooted in the Vasicek model \cite{BN1} which is considered endogenous with time homogeneous parameters (constant over time).  The models extending on the work of Vasicek model have time dependent parameters such as the Hull-White model \cite{HW1}. \\

Short rate models are divided in two categories.  The first one is equilibrium models, Vasicek model and Cox, Ingersoll and Ross (CIR) model are examples of equilibrium short rate models.  The second one is the arbitrage free models, the Ho-Lee model and the Hull-White model are example of arbitrage free models.  We discuss all these models in Section 3. \\ 

The short rate models are either single factor or multi factor, for this paper we will only focus on One-factor short rate models and a two factor short rate model \cite{BA1}.  One factor  models involve a single stochastic factor which is the short rate to determine the future movements of all interest rates.  \\

TODO: Write intro of a two factor model.


\section{Two-Factor Hull-White Model 
}


\section{Comparison with Single Factor Short Rate Models
}

\subsection{Comparison with Vasicek model
}
The Vasicek Model \cite{Vas1} model is one of the earliest stochastic models of the term structure of interest rate.\\

The Vasicek model describes short interest rates as follows,
 \[
dr_t = \alpha(\beta - r_t) dt + \sigma dW_t
\]
Where,\\
$\alpha =$ Speed of mean reversion $0 \leq \alpha \leq 1 $ \\
$\beta =$ Mean reversion level  \\
$r_t$ = Short rate  \\
$\sigma$ = Short rate volatility \\
$W_t$ = Brownian motion process  \\

The model exhibits mean reversion, over time interest rate converge to the mean reversion level $\beta$ at the speed of $\alpha$.  In comparison to the two factor Hull-White model it exhibits several short comings such as there is no term structure volatility, there is only one factor which assumes short rates correlated, it is an endogenous model containing time homogeneous parameters which do not vary over time.


\subsection{Comparison with Cox-Ingersoll-Ross (CIR) model
}

The Cox-Ingersoll-Ross (CIR) model was developed in 1985 \cite{CIR1} by J. C. Cox, J. E. Ingersoll and S. A. Ross.  This model is considered an improvement to the Vasicek model and the square root element does not allow for negative rates.\\

The Cox-Ingersoll-Ross (CIR) model describes short interest rates as follows,
 \[
dr_t = \alpha(\beta - r_t) dt + \sigma \sqrt{r_t} dW_t
\]
Where,\\
$\alpha =$ Speed of mean reversion $0 \leq \alpha \leq 1 $ \\
$\beta =$ Mean reversion level  \\
$r_t$ = Short rate  \\
$\sigma$ = Short rate volatility \\
$W_t$ = Brownian motion process  \\

The model exhibits mean reversion, over time interest rate converge to the mean reversion level $\beta$ at the speed of $\alpha$.  In comparison to the two factor Hull-White model it exhibits several short comings such as there is no term structure volatility, there is only one factor which assumes short rates correlated, it is an endogenous model containing time homogeneous parameters which do not vary over time.

\subsection{Comparison with Ho-Lee model
}
Ho-Lee model developed in 1986 \cite{HoLee1} by Thomas SY Ho and Sang-Bin Lee, it is a single factor short rate model used in pricing of interest rate derivatives.


\subsection{Comparison with Hull-White model
}

\section{Derivative product Payoff}


\section{Conclusion}

In this paper we went through ....................  The purpose of this paper ..............................

\newpage
\section*{} \label{bibsection}


% the second parameter MMMMM should be as long as the longest label you use, in my case Smoller -- if you use % numbers only, use 99
% use \cite{refname} to refer to bibliography item \bibitem{refname} 
% LaTeX assigns a number, unless you use \bibitem[Name]{refname} -- in this case
% LaTeX prints Name when you use \cite{refname}
\begin{thebibliography}{MMMMM} 
\bibitem{BA1} Blanchard, Arnaud. "The two-factor Hull-White model: pricing and calibration of interest rates derivatives." KTH Royal Insitute of Technology (2014).
\bibitem{HoLee1} Ho, Thomas SY, and Sang‐Bin Lee. "Term structure movements and pricing interest rate contingent claims." the Journal of Finance 41.5 (1986): 1011-1029.
\bibitem{Vas1} Vasicek, Oldrich. "An equilibrium characterization of the term structure." Journal of financial economics 5.2 (1977): 177-188.
\bibitem{CIR1} Cox, John C., Jonathan E. Ingersoll Jr, and Stephen A. Ross. "An intertemporal general equilibrium model of asset prices." Econometrica: Journal of the Econometric Society (1985): 363-384.
\bibitem{HW1} Hull, John, and Alan White. "Pricing interest-rate-derivative securities." The review of financial studies 3.4 (1990): 573-592.
\bibitem{HW2} Hull, John, and Alan White. "One-Factor Interest Rate Models and the Valuation of Interest Rate Derivative Securities" Journal of Finance and Quantitative Analysis, volume 28,  (1993): 235 - 254.
\bibitem{HW3} Hull, John, and Alan White. "Numerical procedures for implementing term structure models I: Single-factor models." Journal of derivatives 2.1 (1994): 7-16.
\bibitem{HW4} Hull, John, and Alan White. "The general Hull–White model and supercalibration." Financial Analysts Journal 57.6 (2001): 34-43.
\bibitem{HW5} Hull, John. "Using Hull-White interest rate trees." Journal of derivatives 3.3 (1996): 26-36.
\bibitem{BN1} Burgess, Nicholas, An Overview of the Vasicek Short Rate Model (August 12, 2014). Available at SSRN: https://ssrn.com/abstract=2479671 or http://dx.doi.org/10.2139/ssrn.2479671
\bibitem{HJM1} Heath, David, Robert Jarrow, and Andrew Morton. "Bond pricing and the term structure of interest rates: A new methodology for contingent claims valuation." Econometrica: Journal of the Econometric Society (1992): 77-105.



\end{thebibliography}
\bibliographystyle{plain}
\bibliography{template}

\end{document}
