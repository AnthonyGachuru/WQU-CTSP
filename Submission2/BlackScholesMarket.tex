\documentclass{article}
\usepackage{graphicx}
% stuff from the percent sign to end of line is a comment, ignored by LaTeX

\usepackage{amsmath,amssymb,graphicx} %load extra symbols and environments
\usepackage[margin=1in]{geometry} %set margins
\usepackage{enumerate}
\begin{document}

\nocite{*} % this command forces all references in template.bib to be printed in the bibliography

\title{Pricing Exotic Options}

\author{
  Ansari, Zain Us Sami Ahmed\\
  \texttt{zainussami@gmail.com}
  \and
  Nguyen, Dang Duy Nghia \\
  \texttt{nghia002@e.ntu.edu.sg}  
    \and
  Srivastava, Anchal \\
  \texttt{anchal.sri82@gmail.com}  
        \and
  Gunawan, Handy \\
  \texttt{handygunawan17@gmail.com}  
}

\date{Feb. 20, 2020} % if this is omitted, the current date is used for the title page

\maketitle

\noindent
\textbf{Keywords:} Black-Scholes model, Exotic options,  Lookback options, Barrier options, Compound options
 and Asset Pricing.



% the following creates an abstract -- it can be omitted
% an example of an environment: these have the form \begin{name} ... \end{name}
\begin{abstract}
In this paper we discuss the pricing of exotic options in a Black-Scholes market.  In the first section we introduce the Black-Scholes model in a financial market, in the second section we derive the price of loopback options, in the third section we find the price of barrier options,  in the fourth section we analyse work of Strasbourg school of probability and we conclude by discussing the purpose of this paper.  
\end{abstract}

\section{Introduction
}

Black-Scholes model is a commonly used option pricing model. This model was discovered by the economists Black and Scholes \cite{BS1} in 1973.  The modelling assumptions for this paper are as follows.  The market consists of two assets, the first one is risky asset which is the stock $S$ and the second one is riskless asset the bank account $B$. \\

The riskless asset denoted by $B$ and is modelled by a differential equation 

\[
dB_t=r B_t dt
\]

The risk-free interest rate $r$ is constant.\\

The risky asset denoted by $S$ and is modelled by the following SDE
\[
dS_t= S_t(\mu dt + \sigma dW_t) 
\]

If we apply It\={o}'s lemma, we get.

\[
S_t= S_0 e^{(\mu- \frac{\sigma^2}{2})t  + \sigma W_t} 
\]

\[
B_t= B_0  e^{rt}
\]

Therefore, $S_t$ is a geometric Brownian motion with drift $(\mu- \frac{\sigma^2}{2})$ and volatility $\sigma$.  Additional assumptions include costless trading, constant divided on the risky asset and restrictions on the riskless asset. \\



\section{Lookback options 
}
\subsection{Nature of Lookback options
}

Lookback options are financial derivatives whose payoff depends on the maximum or the minimum price of the underlying assets during a certain period of the life of the option contract. Different schemes of choosing the strike price give different types of lookback options. In this section, we discuss four most common types of lookback options, namely fixed strike call/put options and floating strike call/put options.

For fixed strike options, the strike price is determined at purchase. The payoff for call option is the difference between the maximum asset price during the option life and the predetermined strike price, written in full:
\[
H=max(M_{T_0}^T - K, 0)
\]
where $K$ is the predetermined strike price and $M_{T_0}^T=\max_{T_0 \leq t \leq T}(S_t)$ is the maximum price of the underlying asset during the lookback period. Similarly the payoff of fixed strike put option is written as:
\[
H=max(K - m_{T_0}^T, 0)
\]
with $m_{T_0}^T=\min_{T_0 \leq t \leq T}(S_t)$ being the minimum price of the underlying asset during its lookback period.

For the floating stike counterparts, the strike price is determined automatically at maturity such that it is beneficial for the holder. The payoff for call option is the difference between the asset price at maturity and the minimum asset price during its lookback period. Payoff function for call and put function is given respectively:
\[
H=max(S_T - m_{T_0}^T, 0)
\]

\[
H=max(M_{T_0}^T - S_T, 0)
\]

Given the payoff functions as above, in 2.3 we are going to solve their PDE to derive the price for each of these options in the same manner as pricing Vanilla European options with Black-Scholes PDE.
But first, let us look at some examples of lookback options in real life.
\subsection{Examples of Lookback options
}

In reality, lookback options, as well as other exotic options, are commonly not traded publicly on exchanges but over-the-counter (OTC).
Suppose we bought 2 lookback call options, one with fixed strike \$100 and one with floating strike
on an underlying stock when it was also trading at \$100 with 1 year maturity.
During that one year, the price of the stock goes up to \$150 at ceiling, hits bottom at \$80, and goes back to \$100 at maturity.
When we exercise the contract, for the fixed strike option we gain a profit of \$150 - \$100 = \$50,
since the payoff depends on the most favorable price of the underlying stock.
For floating strike option, since the lowest price of the stock is \$80 and current price of the stock is \$100,
the profit is \$100 - \$80 - \$20.
Lookback options are said to have anti-regret properties, in ways that investors do not have to regret
selling too early or holding on too long to be suddenly swept by a correction. \\

Let's take another example. Consider two same options, with the underlying stock still having \$150 at high
and \$80 at low, but now the stock price at maturity is \$130. The profit for fixed strike option remains the
same in this case at \$50, but the floating strike option makes profit of \$130 - \$80 = \$50. \\

In the third scenario, consider the stock price closes off at its bottom \$80. That sets the strike price at \$80 too,
and yields zero profit for the floating strike option. The profit for fixed strike option does not depend on
the stock price at maturity in case it is still higher than its lowest point during the lookback period.
\subsection{Analytical solution for Lookback options
}
\subsubsection{Fixed strike lookback call option
}
Let $\widetilde{W}$ be a $\mathbb{P}^{\star}$-Brownian motion. As we know, the asset price $S_t$ follows the dynamics:
\[
dS_t=S_t(rdt + \sigma d \widetilde{W_t})
\]
which yields
\[
S_t=S_0 e^{(r - \frac{\sigma^2}{2})t + \sigma \widetilde{W_t}}
\]
Now we calculate the price of the fixed strike call option that we define above at time of purchase $T_0=0$
\begin{align*}
\pi(H) 	&= \mathbb{E^{\star}}[e^{-rT} max(M_{T_0}^T - K, 0)] \\
			&= e^{-rT} \mathbb{E^{\star}}[max(M_{0}^T - K, 0)] \\
			&= e^{-rT} \int_{0}^{\infty} \mathbb{P}^{\star}[M_{0}^T - K \geq x]dx \\
			&= e^{-rT} \int_{0}^{\infty} \mathbb{P}^{\star}[M_{0}^T \geq x+K]dx \\
			&= e^{-rT} \int_{0}^{\infty} \mathbb{P}^{\star}[ln \frac{M_{0}^T}{S_0} \geq ln \frac{x+K}{S_0}]dx \\
			&= e^{-rT} \int_{K}^{\infty} \mathbb{P}^{\star}[ln \frac{M_{0}^T}{S_0} \geq ln \frac{z}{S_0}]dz &(\text{change of variable } z=x+K) \\
			&= e^{-rT} \int_{K}^{\infty} ( 1 - \mathbb{P}^{\star}[ln \frac{M_{0}^T}{S_0} \leq ln \frac{z}{S_0}])dz &(1)
\end{align*}
Now we define a stochastic variable $Y_T= ln \frac{M_{0}^{T}}{S_0}$. By \cite{Peter1},
\begin{align*}
\mathbb{P}^{\star}[Y_T \leq y]= \Phi \left(\frac{y-(r-\frac{\sigma^2}{2})T}{\sigma\sqrt{T}} \right) - e^{\frac{2r}{\sigma^2} - 1} \Phi \left(\frac{-y-(r-\frac{\sigma^2}{2})T}{\sigma\sqrt{T}} \right)
&, y \geq 0
\end{align*}
Let $y=ln\frac{z}{S_0}$, from (1) we have
\begin{align*}
\pi(H) 	&= e^{-rT} \int_{ln\frac{K}{S_0}}^{\infty} S_0e^y( 1 - \mathbb{P}^{\star}[Y_T \leq y])dy \\
			&= e^{-rT}S_0 \int_{ln\frac{K}{S_0}}^{\infty} e^y \left( 1 -  
			\Phi \left(\frac{y-(r-\frac{\sigma^2}{2})T}{\sigma\sqrt{T}} \right) + e^{\frac{2r}{\sigma^2} - 1} \Phi \left(\frac{-y-(r-\frac{\sigma^2}{2})T}{\sigma\sqrt{T}} \right)
			\right) dy \\
			&= e^{-rT}S_0 \int_{ln\frac{K}{S_0}}^{\infty} e^y \left( 
			e^{\frac{2r}{\sigma^2} - 1} \Phi \left(\frac{-y-(r-\frac{\sigma^2}{2})T}{\sigma\sqrt{T}} \right)
			+ \Phi \left(\frac{(r-\frac{\sigma^2}{2})T - y}{\sigma\sqrt{T}} \right)
			\right) dy \\
			&= S_0 \Phi(d) - e^{-rT}K \Phi(d - \sigma \sqrt{T} ) \\
			&\hspace{1cm} + e^{-rT}\frac{\sigma^2}{2r}S_0 \left[
			e^{rT}\Phi(d) - \left(\frac{S_0}{K}\right)^{\frac{-2r}{\sigma^2}} \Phi\left(d- \frac{2r}{\sigma}\sqrt{T}\right)
			\right]
\end{align*}
where $d = \frac{ln\frac{S_0}{K} + (r+\frac{\sigma^2}{2})T}{\sigma\sqrt{T}}$

\subsubsection{Fixed strike lookback put option
}
The fixed strike lookback put option can be priced in similar way.
\begin{align*}
\pi(H) 	&= \mathbb{E^{\star}}[e^{-rT} max(K - m_{0}^T, 0)] \\
			&= e^{-rT} \mathbb{E^{\star}}[max(K - m_{0}^T, 0)] \\
			&= e^{-rT} \int_{0}^{K} \mathbb{P}^{\star}[max(K - m_{0}^T, 0) \geq x]dx  &(0 \leq max(K - m_{t}^{T}, 0) \leq K)\\
			&= e^{-rT} \int_{0}^{K} \mathbb{P}^{\star}[m_{0}^T \leq K-x]dx \\
			&= e^{-rT} \int_{0}^{K} \mathbb{P}^{\star}[ln \frac{m_{0}^T}{S_0} \leq ln \frac{K-x}{S_0}]dx \\
			&= -e^{-rT} \int_{0}^{K} \mathbb{P}^{\star}[ln \frac{m_{0}^T}{S_0} \leq ln \frac{z}{S_0}]dz &(\text{change of variable } z=K-x) \\
			&= -e^{-rT} \int_{0}^{ln\frac{K}{S_0}} S_0e^{y} \mathbb{P}^{\star}[y_T \leq y]dy &(2)
\end{align*}

Again, we define the stochastic variable $y_T= ln \frac{m_{0}^{T}}{S_0}$. By \cite{Peter1},
\begin{align*}
\mathbb{P}^{\star}[y_T \geq y]= \Phi \left(\frac{-y+(r-\frac{\sigma^2}{2})T}{\sigma\sqrt{T}} \right) - e^{\frac{2r}{\sigma^2} - 1} \Phi \left(\frac{y+(r-\frac{\sigma^2}{2})T}{\sigma\sqrt{T}} \right)
&, y \leq 0
\end{align*}
Let $y=ln\frac{z}{S_0}$, from (2) we have
\begin{align*}
\pi(H) 	&= e^{-rT} \int_{0}^{ln\frac{K}{S_0}} S_0e^y( 1 - \mathbb{P}^{\star}[y_T \geq y])dy \\
			&= e^{-rT}S_0 \int_{0}^{ln\frac{K}{S_0}}  e^y \left( 1 -  
			\Phi \left(\frac{-y+(r-\frac{\sigma^2}{2})T}{\sigma\sqrt{T}} \right) + e^{\frac{2r}{\sigma^2} - 1} \Phi \left(\frac{y+(r-\frac{\sigma^2}{2})T}{\sigma\sqrt{T}} \right)
			\right) dy \\
			&= e^{-rT}S_0 \int_{0}^{ln\frac{K}{S_0}} e^y \left( 
			\Phi \left(\frac{y-(r-\frac{\sigma^2}{2})T}{\sigma\sqrt{T}} \right) + e^{\frac{2r}{\sigma^2} - 1} \Phi \left(\frac{y+(r-\frac{\sigma^2}{2})T}{\sigma\sqrt{T}} \right)
			\right) dy \\
			&= -S_0 \Phi(-d) + e^{-rT}K \Phi(-d + \sigma \sqrt{T} ) \\
			&\hspace{1cm} + e^{-rT}\frac{\sigma^2}{2r}S_0 \left[
			\left(\frac{S_0}{K}\right)^{\frac{-2r}{\sigma^2}} \Phi\left(-d+ \frac{2r}{\sigma}\sqrt{T} \right) - e^{rT}\Phi(-d)
			\right]
\end{align*}

\subsubsection{Floating strike lookback call option
}
In the last two sections, we have already done all the heavy lifting of deriving the price equations for fixed strike lookback options.
In the next two sections, we will show that the price of floating strike lookback options can be easily deduced from their fixed counterparts.

Consider first a floating strike call option's price $\pi(H) = \mathbb{E^{\star}}[e^{-rT} max(S_T - m_{0}^T, 0)]$.
It is obvious to see that $S_T \geq m_{T_0}^T$, always, by definition. Hence we can rewrite
\begin{align*}
\pi(H) 	&= \mathbb{E^{\star}}[e^{-rT} (S_T - m_{0}^T)] \\
		&= e^{-rT} \mathbb{E^{\star}}[S_T - m_{0}^T] \\
		&= e^{-rT} \mathbb{E^{\star}}[(S_T - S_0) + max(S_{0} - m_{0}^T, 0)] \\
		&= e^{-rT} \mathbb{E^{\star}}[S_T - S_0] + e^{-rT} \mathbb{E^{\star}}[max(S_{0} - m_{0}^T, 0)] \\
\end{align*}
Now we can see that the second addend is exactly the price of a lookback put option at fixed strike $S_0$.
Substitute the formular in 2.2.2, we have
\begin{align*}
\pi(H)	&= S_0 \Phi(d_0) - e^{-rT}S_0 \Phi(d_0 - \sigma \sqrt{T} ) \\
		&\hspace{1cm} + e^{-rT}\frac{\sigma^2}{2r}S_0 \left[
			\Phi\left(-d_0+ \frac{2r}{\sigma}\sqrt{T} \right) - e^{rT}\Phi(-d_0)
		\right]
\end{align*}
where $d_0 = d|_{K = S_0} = \frac{(r+\frac{\sigma^2}{2})T}{\sigma\sqrt{T}} = (\frac{r}{\sigma} + \frac{\sigma}{2})\sqrt{T}$
\subsubsection{Floating strike lookback put option
}
Using the same logic to relate the price of a floating strike lookback put option to the price of a fixed strike lookback call option.
We have
\begin{align*}
\pi(H) 	&= -S_0 \Phi(-d_0) + e^{-rT}S_0 \Phi(-d_0 + \sigma \sqrt{T} ) \\
		&\hspace{1cm} + e^{-rT}\frac{\sigma^2}{2r}S_0 \left[
			e^{rT}\Phi(d_0) - \Phi\left(d_0 - \frac{2r}{\sigma}\sqrt{T} \right)
		\right]
\end{align*}
\section{Barrier options
}


\[
dX_t=\sigma(X_t)dW_t + \mu(X_t) dt
\]


\section{Compound options
}
A compound option is a standard option for which the underlying asset is another standard option. Geske \cite{Geske1} suggests this option can be generalized to value many corporate liabilities.  Since the overlying and underlying are both options there are two strike prices and two exercise dates. 

\subsection{Examples of Compound options
}
Speculation remain the biggest real world use case of compound options but enterprises and businesses can really benefit from compound options.\\

In the first scenario, compound options as a mean for contingency hedging. A Singaporean IT firm is bidding for a project with the United States government to start in 6 months, upon completion of the project the Singaporean IT firm will earn 2 million USD 12 months after the start of the contract.  The IT firm needs to hedge its dollar exposure against a possibility of a lower USD/SGD rate.  If they purchase a call option on USD-put option.  If the bid is successful and USD-put rises in value they exercise the compound option, if they USD-put falls in value they buy the USD-put in the market.  In the case where the company is not successful in the bid and USD-put rises in value they exercise the compound option and if the USD-put falls in value the compound option expires.  In all the cases, the loss to the IT firm is limited to premium paid for the compound option.\\

In the second scenario, a large telecommunication service provider is bidding for a spectrum license.  If they win the bid they will need to borrow 100 million dollars for two years for equipment upgrade and if they do not win they will be left 100 million dollars of financing they do not need.  In this case, the best option is to buy a compound option.  Since all the calculations the company did is on current interest rate they need to buy an interest rate cap, the service provider can buy a call option on a two year interest rate cap.  So now, if they win the bid they can exercise option for a two year cap and in the scenario they do not win the bid they let the option expire because underlying financing is not needed any more. \\

\section{Conclusion}
In this paper we went through derivations of pricing formulas for Lookback Options, Barrier Options and Compound Options using Black Scholes model.  We also presented brief examples of real world use cases of each of the options discussed.  The purpose of this paper was to discuss the set of tools to value exotic options.

\newpage
\section*{} \label{bibsection}


% the second parameter MMMMM should be as long as the longest label you use, in my case Smoller -- if you use % numbers only, use 99
% use \cite{refname} to refer to bibliography item \bibitem{refname} 
% LaTeX assigns a number, unless you use \bibitem[Name]{refname} -- in this case
% LaTeX prints Name when you use \cite{refname}
\begin{thebibliography}{MMMMM} 
\bibitem{BS1} Black, Fischer; Myron Scholes (1973). "The Pricing of Options and Corporate Liabilities". Journal of Political Economy.
\bibitem{BS2} Merton, Robert C. (1973). "Theory of Rational Option Pricing". Bell Journal of Economics and Management Science.
\bibitem{HJ1} Hull, John C. (1997). Options, Futures, and Other Derivatives. Prentice Hall.
\bibitem{Peter1} Zhang, Peter G. (1998). Exotic Options: A Guide to Second Generation Options. World Scientific Publishing.
\bibitem{Geske1} Geske, R (1979). The Valuation of Compound Options, Journal of Financial Mathematics.
\bibitem{Rub1} Rubinstein, M (1991). Double Trouble.
\bibitem{Rub2} Cox,J. and M. Rubinstein (1985). Options Markets. Prentice Hall.
\bibitem{Rub3} Rubinstein, M (1991). Pay now, Choose later.
\bibitem{Rub4} E.Reiner and M. Rubinstein (1991). Breaking down the Barriers. Prentice Hall.
\bibitem{Rei1} Reiner, E (1992). Quanto Mechanics.


\end{thebibliography}
\bibliographystyle{plain}
\bibliography{template}

\end{document}
